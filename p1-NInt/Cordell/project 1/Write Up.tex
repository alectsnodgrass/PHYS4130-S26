\documentclass[11pt,letterpaper]{article}
\usepackage[T1]{fontenc}
\usepackage[utf8]{inputenc}
\usepackage{lmodern}
\usepackage[margin=1in]{geometry}
\usepackage{microtype}
\usepackage{setspace}
\onehalfspacing
\usepackage{parskip}
\usepackage{amsmath,amssymb,amsfonts}
\usepackage{siunitx}
\sisetup{separate-uncertainty=true}
\usepackage{mhchem}
\usepackage{graphicx}
\usepackage[font=small,labelfont=bf]{caption}
\usepackage{subcaption}
\usepackage{booktabs}
\usepackage{float} 
\usepackage[hidelinks]{hyperref}
\usepackage[capitalise,nameinlink]{cleveref}
\usepackage[backend=biber,style=ieee,sorting=nyt]{biblatex}

\begin{document}

\section{Attribution}
attribtuions

\section{Timekeeping}
Approx values

Sunday: 2 hours\\
I defined most of the functions I would need for the project and implemented the Trapezoid rule. Then I set up LaTeX.

Monday: 0 hours

Tuesday: 1.25 hours\\
I worked on figuring out how to use github.

Wednesday: 2.5 hours\\
I implemented the Gaussian quadrature and plotted the Legendre polynomials.

Thursday: 1 hour
Write up.

\section{Languages, Libraries, Lessons Learned}
I wrote all of my code in python from start to finish. I tried my best to mostly work in code.py 
so it would be easier to interpret my github commits, but I also used Code\_NB.py for the quality 
of life that a notebook offers when it comes to things like debugging and running a segment of 
code at a time. I also implemented various math functions from the numpy library such as np.sin, 
np.abs (absolute value), and np.sum. I used pylab for plotting. Both of these libraries are pretty 
standard because of their usefulness and simplicity, and I was already fairly familiar with them. 
In addition, I used the scipy library as instructed for Legendre polynomials. I would consider this 
pretty remarkable as I did not have to calculate them myself which would be a pain. Meanwhile, 
numpy and pylab are pretty remarkable in their own way--- they come in handy very often.

\end{document}